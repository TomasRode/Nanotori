\documentclass[a4paper, 12 pt]{article}
\usepackage[utf8]{inputenc}
\usepackage[T1]{fontenc}
\usepackage[slovene]{babel}
\usepackage{lmodern}




\title{Nanotori}
\author{Tomas Rode \\ Enej Kovač}

\begin{document}
\maketitle

\section{Navodila}

Nanotor je 3-regularen graf na torusu. Vsak nanotor lahko dobimo tako, da na šeskotni mreži enačimo nasprotni stranici danega paralelograma. Torej je vsak nanotorus določen z dvema vektorjema v ravnini ${(k,l)$ in $(m, n)}$ za katere velja ter $k^2 + l^2 != 0$ in $m^2 + n^2 != 0$.

\section{Opis dela}

Za izvedbo naloge sva se odločila za programski jezik \textit{Python}, v katerega sva za lažje delo z grafi naložila knjižnice \texttt{networkx, numpy, matplotlib, random} in \texttt{time}.
\textit{Tukaj sledi še KRATEK opis, kaj koda dela}.

\section{Načrt za nadaljnje delo}

Ko bova na podlagi dovolj velikega vzorca dreves videla, katera so za najina problema najbolj ugodna, bova s pomočjo genetskega algoritma postopoma izbirala le tista najugodnejša in jih na vsakem koraku križala ali modificirala. Ta postopek bova ponavljala, dokler bo postopek na posameznem koraku še vračal ugodnejša drevesa.



\end{document}
