\documentclass[a4paper, 12 pt]{article}
\usepackage[utf8]{inputenc}
\usepackage[T1]{fontenc}
\usepackage[slovene]{babel}
\usepackage{lmodern}
\usepackage{graphix}



\title{Nanotori}
\author{Tomas Rode \\ Enej Kovač}

\begin{document}
\maketitle

\section{Navodila}

Nanotor je 3-regularen graf na torusu. Vsak nanotor lahko dobimo tako, da na šeskotni mreži enačimo nasprotni stranici danega paralelograma. Torej je vsak nanotorus določen z dvema vektorjema v ravnini $(k,l)$ in $(m, n)$ za katere velja $k^2 + l^2 \neq 0$ in $m^2 + n^2 \neq 0$. Projekt je sestavljen iz štirih podnalog:

\begin{itemize}
  \item V prvem delu naloge morava ustvariti funkcijo, ki v \textit{Sage} konstruira nanotor, za dane $k, l, m, n$.
  \item V drugem delu morava s pomočjo funkcij v \textit{Sage} preučiti nekaj lastnosti nanotorov: za dane $k, l, m, n$ morava določiti število vozlišč, premer, tranzitivnost in druge lastnosti nanotora.
  \item Tretji del naloge zahteva, da za $v \in V(T)$, poljubno vozlišče nanotora, določiva število vozlišč na razdaljah $i,\ 1 \leq i \leq \text{diam}(T)$. Poiskati morava formulo za dane $i, k, l, m, n$.
  \item Naj bo $T$ nanotorus tipa $(k, l, m, n)$. Ugotoviti morava ali obstaja nanotorus tipa $(k', 0, m', n')$, izomorfen T. Če obstaja, naloga zahteva, da ugotoviva odnos med $k, l, m, n$ in $k', m', n'$.
\end{itemize}

\section{Opravljeno delo (do 5. 11. 2019)}

S pomočjo objektnega programiranja sva v \textit{Sage} zapisala funkcijo, ki konstruira nanotor s $k, l, m$ in $n$. Na sponjih slikah lahko vidimo nekaj primerov.

\section{Načrt za nadaljnje delo}

Delo bova nadaljevala pri nasledjih točkah iz navodil, torej bova najprej ugotavljala lastnosti nanotorov. Iz teh ugotovitev bova potem preverjala, kakšne so razdalje v nanotoru in kdaj sta nanotora izomorfna.


\end{document}
